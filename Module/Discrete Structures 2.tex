%%%%%%%%%%%%%%%%%%%%%%%%%%%%%%%%%%%%%%%%%%%%%%%%%%%
%% LaTeX book template                           %%
%% Author:  Amber Jain (http://amberj.devio.us/) %%
%% License: ISC license                          %%
%%%%%%%%%%%%%%%%%%%%%%%%%%%%%%%%%%%%%%%%%%%%%%%%%%%

\documentclass[a4paper,11pt,oneside]{book}
\usepackage{modulestyle}

%%%%%%%%%%%%%%%%%%%%%%%%%%%%%%%%%%%%%%%%%%%%%%%%%%%%%%%%%
% Source: http://en.wikibooks.org/wiki/LaTeX/Hyperlinks %
%%%%%%%%%%%%%%%%%%%%%%%%%%%%%%%%%%%%%%%%%%%%%%%%%%%%%%%%%

%%%%%%%%%%%%%%%%%%%%%%%%%%%%%%%%%%%%%%%%%%%%%%%%%%%%%%%%%%%%%%%%%%%%%%%%%%%%%%%%
% 'dedication' environment: To add a dedication paragraph at the start of book %
% Source: http://www.tug.org/pipermail/texhax/2010-June/015184.html            %
%%%%%%%%%%%%%%%%%%%%%%%%%%%%%%%%%%%%%%%%%%%%%%%%%%%%%%%%%%%%%%%%%%%%%%%%%%%%%%%%
\newenvironment{dedication}
{
   \cleardoublepage
   \thispagestyle{empty}
   \vspace*{\stretch{1}}
   \hfill\begin{minipage}[t]{0.66\textwidth}
   \raggedright
}
{
   \end{minipage}
   \vspace*{\stretch{3}}
   \clearpage
}

%%%%%%%%%%%%%%%%%%%%%%%%%%%%%%%%%%%%%%%%%%%%%%%%
% Chapter quote at the start of chapter        %
% Source: http://tex.stackexchange.com/a/53380 %
%%%%%%%%%%%%%%%%%%%%%%%%%%%%%%%%%%%%%%%%%%%%%%%%
\makeatletter
\renewcommand{\@chapapp}{}% Not necessary...
\newenvironment{chapquote}[2][2em]
  {\setlength{\@tempdima}{#1}%
   \def\chapquote@author{#2}%
   \parshape 1 \@tempdima \dimexpr\textwidth-2\@tempdima\relax%
   \itshape}
  {\par\normalfont\hfill--\ \chapquote@author\hspace*{\@tempdima}\par\bigskip}
\makeatother

%%%%%%%%%%%%%%%%%%%%%%%%%%%%%%%%%%%%%%%%%%%%%%%%%%%
% First page of book which contains 'stuff' like: %
%  - Book title, subtitle                         %
%  - Book author name                             %
%%%%%%%%%%%%%%%%%%%%%%%%%%%%%%%%%%%%%%%%%%%%%%%%%%%

\newcommand{\BookTitle}{Discrete Structures 2}
\newcommand{\BookTitleFootnote}{A course in the Bachelor of Science in Computer
Science}

\newcommand{\BookSubtitle}{A Study Guide for Students of Sorsogon State University - Bulan Campus}
\newcommand{\BookSubtitleFootnote}{This book is a study guide for students of
Sorsogon State University - Bulan Campus taking up the course Discrete Structures 2.}

\newcommand{\BookAuthorFirstName}{Jarrian Vince}
\newcommand{\BookAuthorLastName}{Gojar}
\newcommand{\BookAuthorName}{Jarrian Vince G. Gojar}
\newcommand{\BookAuthorURL}{https://github.com/godkingjay}

% Book's title and subtitle
\title{\Huge \textbf{\BookTitle}  \footnote{\BookTitleFootnote} \\
\huge \BookSubtitle \footnote{\BookSubtitleFootnote}}

% Author
\author{\textsc{\BookAuthorName}\thanks{\url{\BookAuthorURL}}}

\begin{document}

\frontmatter
\maketitle

%%%%%%%%%%%%%%%%%%%%%%%%%%%%%%%%%%%%%%%%%%%%%%%%%%%%%%%%%%%%%%%
% Add a dedication paragraph to dedicate your book to someone %
%%%%%%%%%%%%%%%%%%%%%%%%%%%%%%%%%%%%%%%%%%%%%%%%%%%%%%%%%%%%%%%
\begin{dedication}
Sorsogon State University - Bulan Campus
\end{dedication}

%%%%%%%%%%%%%%%%%%%%%%%%%%%%%%%%%%%%%%%%%%%%%%%%%%%%%%%%%%%%%%%%%%%%%%%%
% Auto-generated table of contents, list of figures and list of tables %
%%%%%%%%%%%%%%%%%%%%%%%%%%%%%%%%%%%%%%%%%%%%%%%%%%%%%%%%%%%%%%%%%%%%%%%%
\tableofcontents
\listoffigures
\listoftables
% \lstlistoflistings

\mainmatter

%%%%%%%%%%%
% Preface %
%%%%%%%%%%%
\chapter*{Preface}
\begin{chapquote}{John von Neumann}
  ``If people do not believe that mathematics is simple, it is only
  because they do not realize how complicated life is.''
  \end{chapquote}

\noindent \BookAuthorName \\
\noindent \url{\BookAuthorURL}

%%%%%%%%%%%%%%%%%%%%%%%
%%%     Chapter     %%%
%%%%%%%%%%%%%%%%%%%%%%%
\chapter{Boolean Algebra}

\section{Introduction}

Circuits in computers are made up of millions of tiny switches that
can be in one of two states: on or off. These switches are controlled
by electrical signals that represent logical values. The behavior of
these switches can be described using a mathematical system called
Boolean Algebra. \textbf{Boolean Algebra} is a branch of mathematics
that deals with logical values and operations on these values. It is 
widely used in computer science and engineering to design and analyze
digital circuits. Computers uses the binary number system, which has
only two digits: $0$ and $1$ which means ``low voltage'' and ``high
volt'' respectively. These digits correspond to the logical
values \textcolor{red}{FALSE} and \textcolor{ForestGreen}{TRUE},
respectively.

\section{History of Boolean Algebra}

\textbf{George Boole} was an English mathematician and logician who lived
in the 19th century. He was born in 1815 and died in 1864. Boole is best
known for his work in the field of logic, which laid the foundation for
modern computer science.

Boole's most famous work is his book \textit{The Laws of Thought}, which
was published in 1854. In this book, Boole introduced the concept of
Boolean Algebra, which is a mathematical system for dealing with logical
values. Boolean Algebra is based on the idea that logical values can be
represented as either \textcolor{ForestGreen}{TRUE} or \textcolor{red}{FALSE}.

In 1938, \textbf{Claude Shannon} showed that the two-valued Boolean
Algebra could be used to describe the operation of electrical switches.
This discovery laid the foundation for the design of digital circuits
and computers.

\section{Fundamental Operations}

The three fundamental operations of Boolean Algebra are:

\begin{itemize}
\item \textbf{AND} - The AND operation takes two or more inputs and produces a
  $1$ output only if all inputs are $1$.
\item \textbf{OR} - The OR operation takes two or more inputs and produces a
  $1$ output if at least one input is $1$.
\item \textbf{NOT} - The NOT operation takes a single input and produces
  the opposite value. If the input is $1$, the output is $0$, and vice versa.
\end{itemize}

\begin{table}[!h]
  \centering
  \begin{tabular}{c c c c}
    \hline
    \textbf{} & \textbf{Formal Logic} & \textbf{Set Theory} & \textbf{Boolean Algebra} \\
    \hline
    \textbf{Variables} & p, q, r, ... & A, B, C, ... & x, y, z, ... \\
    
    \textbf{Operations} & $\land$, $\lor$, $\lnot$ & $\cap$, $\cup$, $-$ & •, +, ' \\
    
    \textbf{Special Elements} & $F$, $T$ & $\emptyset$, U & 0, 1 \\
    \hline
  \end{tabular}
  \caption{Comparison of Formal Logic, Set Theory, and Boolean Algebra}
  \label{table:Comparison-Formal-Logic-Set-Theory-Boolean-Algebra}
\end{table}

Table \ref{table:Comparison-Formal-Logic-Set-Theory-Boolean-Algebra} shows
a comparison of the notation used in formal logic, set theory, and Boolean
Algebra. In formal logic, variables are represented by letters such as
$p$, $q$, $r$, etc., and the logical operations are represented by symbols
such as $\land$, $\lor$, and $\lnot$. In set theory, variables are
represented by capital letters such as $A$, $B$, $C$, etc., and the set
operations are represented by symbols such as $\cap$, $\cup$, and $-$. In
Boolean Algebra, variables are represented by letters such as $x$, $y$, $z$,
etc., and the Boolean operations are represented by symbols such as $\cdot$,
$+$, and $'$. The special elements in each system are $F$ and $T$ in formal
logic, $\emptyset$ and $U$ in set theory, and $0$ and $1$ in Boolean Algebra.

Though the notation used in each system is different, the underlying
concepts are the same. For example, the AND operation in Boolean Algebra
is similar to the logical conjunction operation in formal logic, where the
output is \textcolor{ForestGreen}{TRUE} only if all inputs are
\textcolor{ForestGreen}{TRUE}. Similarly, the OR operation in Boolean
Algebra is similar to the logical disjunction operation in formal logic,
where the output is \textcolor{ForestGreen}{TRUE} if at least one input is
\textcolor{ForestGreen}{TRUE}. Compared to set theory, the AND operation
in Boolean Algebra is similar to the intersection operation, where the
output is the set of elements that are common to all input sets.

\subsection{AND Operation}

The AND operation is denoted by the symbol $\cdot$ or juxtaposition. The
output of the AND operation is $1$ only if all inputs are $1$. In Boolean
Algebra, the AND operation is represented by the multiplication symbol
$\cdot$ or by juxtaposition. \textbf{Juxtaposition} is the act of placing
two or more things side by side or close together.

\begin{table}[!h]
  \centering
  \begin{tabular}{c c c}
    \hline
    \textbf{Input 1} & \textbf{Input 2} & \textbf{Output} \\
    \hline
    $0$ & $0$ & $0$ \\
    $0$ & $1$ & $0$ \\
    $1$ & $0$ & $0$ \\
    $1$ & $1$ & $1$ \\
    \hline
  \end{tabular}
  \caption{Truth Table for the AND Operation}
  \label{table:Truth-Table-AND-Operation}
\end{table}

Table \ref{table:Truth-Table-AND-Operation} shows the truth table for the
AND operation. The output is $1$ only if both inputs are $1$; otherwise,
the output is $0$.

Suppose we have the variables $x$ and $y$, and we want to represent the
AND operation between them. We can write this as $x \cdot y$ or $xy$ via
juxtaposition. The output of this operation is $1$ only if both $x$
and $y$ are $1$.

\newpage

\begin{table}[!h]
  \centering
  \begin{tabular}{c c c}
    \hline
    \textbf{Input 1} & \textbf{Input 2} & \textbf{Output} \\
    \hline
    $x$ & $y$ & $xy$ \\
    \hline
    $0$ & $0$ & $0$ \\
    $0$ & $1$ & $0$ \\
    $1$ & $0$ & $0$ \\
    $1$ & $1$ & $1$ \\
    \hline
  \end{tabular}
  \caption{Truth Table for the AND Operation with Variables}
  \label{table:Truth-Table-AND-Operation-Variables}
\end{table}

Table \ref{table:Truth-Table-AND-Operation-Variables} shows the truth
table for the AND operation with variables $x$ and $y$. The output is
$1$ only if both $x$ and $y$ are $1$; otherwise, the output is $0$.

\noindent\fbox{%
  \parbox{\textwidth}{
    \hspace{0.5em}
    
    \textbf{Exercise}

    \begin{enumerate}
      \item Consider the three input AND operation $xyz$. Write
      the truth table for this operation and determine the output for
      each combination of inputs.
    \end{enumerate}

    \hspace{0.5em}
  }
}

\subsection{OR Operation}

The OR operation is denoted by the symbol $+$. The output of
the OR operation is $1$ if at least one input is $1$. In Boolean
Algebra, the OR operation is represented by the addition symbol $+$.

\begin{table}[!h]
  \centering
  \begin{tabular}{c c c}
    \hline
    \textbf{Input 1} & \textbf{Input 2} & \textbf{Output} \\
    \hline
    $0$ & $0$ & $0$ \\
    $0$ & $1$ & $1$ \\
    $1$ & $0$ & $1$ \\
    $1$ & $1$ & $1$ \\
    \hline
  \end{tabular}
  \caption{Truth Table for the OR Operation}
  \label{table:Truth-Table-OR-Operation}
\end{table}

Table \ref{table:Truth-Table-OR-Operation} shows the truth table for the
OR operation. The output is $1$ if at least one input is $1$; otherwise,
the output is $0$.

Suppose we have the variables $x$ and $y$, and we want to represent the
OR operation between them. We can write this as $x + y$. The output of
this operation is $1$ if at least one of $x$ and $y$ is $1$.

\begin{table}[!h]
  \centering
  \begin{tabular}{c c c}
    \hline
    \textbf{Input 1} & \textbf{Input 2} & \textbf{Output} \\
    \hline
    $x$ & $y$ & $x + y$ \\
    \hline
    $0$ & $0$ & $0$ \\
    $0$ & $1$ & $1$ \\
    $1$ & $0$ & $1$ \\
    $1$ & $1$ & $1$ \\
    \hline
  \end{tabular}
  \caption{Truth Table for the OR Operation with Variables}
  \label{table:Truth-Table-OR-Operation-Variables}
\end{table}

Table \ref{table:Truth-Table-OR-Operation-Variables} shows the truth
table for the OR operation with variables $x$ and $y$. The output is
$1$ if at least one of $x$ and $y$ is $1$; otherwise, the output is $0$.

\noindent\fbox{%
  \parbox{\textwidth}{
    \hspace{0.5em}
    
    \textbf{Exercise}

    Write the truth table for the following OR operations:

    \begin{enumerate}
      \item $f(x, y, z) = x + y + z$
      \item $f(x, y, z) = (x + y)z$
      \item $f(x, y, z) = x + yz$
      \item $f(x, y, z) = (x + y)(y + z)$
      \item $f(x, y, z) = x(xy + yz)$
    \end{enumerate}

    \hspace{0.5em}
  }
}

\subsection{NOT Operation}

The NOT operation is denoted by the symbol $'$. The output of the NOT
operation is the opposite of the input. If the input is $1$, the output
is $0$, and vice versa. In Boolean Algebra, the NOT operation is
represented by the prime symbol $'$ or by an overline.

\begin{table}[!h]
  \centering
  \begin{tabular}{c c}
    \hline
    \textbf{Input} & \textbf{Output} \\
    \hline
    $0$ & $1$ \\
    $1$ & $0$ \\
    \hline
  \end{tabular}
  \caption{Truth Table for the NOT Operation}
  \label{table:Truth-Table-NOT-Operation}
\end{table}

Table \ref{table:Truth-Table-NOT-Operation} shows the truth table for the
NOT operation. The output is the opposite of the input. If the input is
$1$, the output is $0$, and vice versa.

Suppose we have the variable $x$, and we want to represent the NOT
operation on it. We can write this as $x'$ or $\overline{x}$. The output
of this operation is the opposite or complement of $x$.

\begin{table}[!h]
  \centering
  \begin{tabular}{c c}
    \hline
    \textbf{Input} & \textbf{Output} \\
    \hline
    $x$ & $x'$ \\
    \hline
    $0$ & $1$ \\
    $1$ & $0$ \\
    \hline
  \end{tabular}
  \caption{Truth Table for the NOT Operation with Variables}
  \label{table:Truth-Table-NOT-Operation-Variables}
\end{table}

Table \ref{table:Truth-Table-NOT-Operation-Variables} shows the truth
table for the NOT operation with variable $x$. The output is the opposite
of $x$. If $x$ is $1$, the output is $0$; if $x$ is $0$, the output is $1$.

\noindent\fbox{%
  \parbox{\textwidth}{
    \hspace{0.5em}
    
    \textbf{Exercise}

    Write the truth table for the following NOT operations:

    \begin{enumerate}
      \item $f(x) = (x')'$
      \item $f(x, y) = (x + y)'$
      \item $f(x, y) = (x \cdot y)'$
      \item $f(x, y, z) = (x + yz)'$
      \item $f(x, y, z) = (x \cdot y + z)'$
      \item $f(x, y, z) = (x + y)'z$
      \item $f(x, y, z) = x' + y' + z'$
      \item $f(x, y, z) = (x + y + z)'$
      \item $f(x, y, z) = x'yz$
      \item $f(x, y) = x' + y'(xy + x')$
    \end{enumerate}

    \hspace{0.5em}
  }
}

\section{Other Operations}

In addition to the fundamental operations of AND, OR, and NOT, there are
several other operations in Boolean Algebra that are commonly used. These
operations include:

\begin{itemize}
  \item \textbf{XOR} - The XOR operation takes two inputs and produces a
    $1$ output if the inputs are different.
  \item \textbf{NAND} - The NAND operation is the complement of the AND
    operation. The output of the NAND operation is $0$ only if all inputs
    are $1$.
  \item \textbf{NOR} - The NOR operation is the complement of the OR
    operation. The output of the NOR operation is $0$ if at least one input
    is $1$.
  \item \textbf{XNOR} - The XNOR operation is the complement of the XOR
    operation. The output of the XNOR operation is $1$ if the inputs are
    the same.
\end{itemize}

\subsection{XOR Operation}

The XOR operation is denoted by the symbol $\oplus$. The output of the XOR
operation is $1$ if the inputs are different. In Boolean Algebra, the XOR
operation is represented by the symbol $\oplus$. If the XOR operation takes
more than two inputs, it is called the \textbf{parity function}. A parity
function is a function that determines whether the number of inputs that
are $1$ is even or odd.

\begin{table}[!h]
  \centering
  \begin{tabular}{c c c}
    \hline
    \textbf{Input 1} & \textbf{Input 2} & \textbf{Output} \\
    \hline
    $0$ & $0$ & $0$ \\
    $0$ & $1$ & $1$ \\
    $1$ & $0$ & $1$ \\
    $1$ & $1$ & $0$ \\
    \hline
  \end{tabular}
  \caption{Truth Table for the XOR Operation}
  \label{table:Truth-Table-XOR-Operation}
\end{table}

Table \ref{table:Truth-Table-XOR-Operation} shows the truth table for the
XOR operation. The output is $1$ if the inputs are different; otherwise,
the output is $0$.

Suppose we have the variables $x$ and $y$, and we want to represent the
XOR operation between them. We can write this as $x \oplus y$. The output
of this operation is $1$ if $x$ and $y$ are different.

\begin{table}[!h]
  \centering
  \begin{tabular}{c c c}
    \hline
    \textbf{Input 1} & \textbf{Input 2} & \textbf{Output} \\
    \hline
    $x$ & $y$ & $x \oplus y$ \\
    \hline
    $0$ & $0$ & $0$ \\
    $0$ & $1$ & $1$ \\
    $1$ & $0$ & $1$ \\
    $1$ & $1$ & $0$ \\
    \hline
  \end{tabular}
  \caption{Truth Table for the XOR Operation with Variables}
  \label{table:Truth-Table-XOR-Operation-Variables}
\end{table}

Table \ref{table:Truth-Table-XOR-Operation-Variables} shows the truth
table for the XOR operation with variables $x$ and $y$. The output is $1$
if $x$ and $y$ are different; otherwise, the output is $0$.

\noindent\fbox{%
  \parbox{\textwidth}{
    \hspace{0.5em}
    
    \textbf{Exercise}

    Write the truth table for the following XOR operations:

    \begin{enumerate}
      \item $f(x, y, z) = (x \oplus y)z$
      \item $f(x, y, z) = x \oplus yz$
      \item $f(x, y, z) = (x \oplus y)(y \oplus z)(z \oplus x)$
      \item $f(x, y, z) = x' \oplus y \oplus z'$
    \end{enumerate}

    \hspace{0.5em}
  }
}

\subsection{NAND Operation}

The NAND operation is simply the complement of the AND operation.
The output of the NAND operation is $0$ only if all inputs are $1$.
In Boolean Algebra, the NAND operation is represented by putting
an overline or $'$ over the AND operation.

\begin{table}[!h]
  \centering
  \begin{tabular}{c c c}
    \hline
    \textbf{Input 1} & \textbf{Input 2} & \textbf{Output} \\
    \hline
    $x$ & $y$ & $(xy)'$ \\
    \hline
    $0$ & $0$ & $1$ \\
    $0$ & $1$ & $1$ \\
    $1$ & $0$ & $1$ \\
    $1$ & $1$ & $0$ \\
    \hline
  \end{tabular}
  \caption{Truth Table for the NAND Operation with Variables}
  \label{table:Truth-Table-NAND-Operation-Variables}
\end{table}

Table \ref{table:Truth-Table-NAND-Operation-Variables} shows the truth
table for the NAND operation with variables $x$ and $y$. The output is
$0$ only if both $x$ and $y$ are $1$; otherwise, the output is $1$.

\subsection{NOR Operation}

The NOR operation is simply the complement of the OR operation. The
output of the NOR operation is $0$ if at least one input is $1$. In
Boolean Algebra, the NOR operation is represented by putting an overline
or $'$ over the OR operation.

\begin{table}[!h]
  \centering
  \begin{tabular}{c c c}
    \hline
    \textbf{Input 1} & \textbf{Input 2} & \textbf{Output} \\
    \hline
    $x$ & $y$ & $(x + y)'$ \\
    \hline
    $0$ & $0$ & $1$ \\
    $0$ & $1$ & $0$ \\
    $1$ & $0$ & $0$ \\
    $1$ & $1$ & $0$ \\
    \hline
  \end{tabular}
  \caption{Truth Table for the NOR Operation with Variables}
  \label{table:Truth-Table-NOR-Operation-Variables}
\end{table}

Table \ref{table:Truth-Table-NOR-Operation-Variables} shows the truth
table for the NOR operation with variables $x$ and $y$. The output is
$0$ if at least one of $x$ and $y$ is $1$; otherwise, the output is $1$.

\subsection{XNOR Operation}

The XNOR operation is simply the complement of the XOR operation. The
output of the XNOR operation is $1$ if the inputs are the same. In
Boolean Algebra, the XNOR operation is represented by putting an overline
or $'$ over the XOR operation.

\begin{table}[!h]
  \centering
  \begin{tabular}{c c c}
    \hline
    \textbf{Input 1} & \textbf{Input 2} & \textbf{Output} \\
    \hline
    $x$ & $y$ & $(x \oplus y)'$ \\
    \hline
    $0$ & $0$ & $1$ \\
    $0$ & $1$ & $0$ \\
    $1$ & $0$ & $0$ \\
    $1$ & $1$ & $1$ \\
    \hline
  \end{tabular}
  \caption{Truth Table for the XNOR Operation with Variables}
  \label{table:Truth-Table-XNOR-Operation-Variables}
\end{table}

Table \ref{table:Truth-Table-XNOR-Operation-Variables} shows the truth
table for the XNOR operation with variables $x$ and $y$. The output is
$1$ if $x$ and $y$ are the same; otherwise, the output is $0$.

\section{Tautology and Fallacy}

In Boolean Algebra, a \textbf{tautology} is a statement that is always
\textcolor{ForestGreen}{TRUE}, regardless of the values of its variables.
A \textbf{fallacy} is a statement that is always \textcolor{red}{FALSE},
regardless of the values of its variables.

\begin{table}[!h]
  \centering
  \begin{tabular}{c c c c}
    \hline
    \textbf{$x$} & \textbf{$x'$} & \textbf{$x + x'$} & \textbf{$xx'$} \\
    \hline
    $0$ & $1$ & $1$ & $0$ \\
    $1$ & $0$ & $1$ & $0$ \\
    \hline
  \end{tabular}
  \caption{Examples of Tautologies and Fallacies}
  \label{table:Examples-Tautologies-Fallacies}
\end{table}

Table \ref{table:Examples-Tautologies-Fallacies} shows examples of
tautologies and fallacies. The expression $x + x'$ is a tautology because
it is always \textcolor{ForestGreen}{TRUE}, regardless of the value of
$x$. The expression $xx'$ is a fallacy because it is always
\textcolor{red}{FALSE}, regardless of the value of $x$.

\section{Boolean Functions}

A \textbf{Boolean function} is a function that takes one or more Boolean
variables as input and produces a Boolean output. Boolean functions are
used to represent logical operations in Boolean Algebra. The output of a
Boolean function is determined by the values of its input variables and
the logical operations applied to them. It is also known as a 
\textbf{switching function}. Boolean variables are variables that can
take on one of two values: $0$ or $1$. In Boolean Algebra, variables
are typically denoted by letters such as $x$, $y$, $z$, etc. The
values of these variables represent logical values: $0$ corresponds
to \textcolor{red}{FALSE}, and $1$ corresponds to
\textcolor{ForestGreen}{TRUE}.

A Boolean function is a function in the form $f: B^n \rightarrow B$,
where $B = \{0, 1\}$ is the set of Boolean values, and $n$ is the
number of input variables and is called the \textbf{arity} of the
function.

A \textbf{literal} is a variable or its complement. For example, $x$
and $x'$ are literals. In the boolean function $f = (x + yz) + x'$,
there are three variables: $x$, $y$, and $z$. The literals in this
function are $x$, $y$, $z$, and $x'$ which are the variables and
their complements.

% Exercises
\noindent\fbox{%
  \parbox{\textwidth}{
    \hspace{0.5em}
    
    \textbf{Exercises}

    Write the truth table for the following Boolean functions:

    \begin{enumerate}
      \item $f(x, y) = [xy + (x + y)']'$
      \item $f(x, y) = (x + y) \oplus (xy)'$
      \item $f(x, y, z) = x(y + z')$
      \item $f(x, y, z) = (x + y)(y + z)(z + x)$
      \item $f(x, y, z) = x \oplus y \oplus z$
    \end{enumerate}

    \hspace{0.5em}
  }
}

\section{Laws of Boolean Algebra}

The laws of Boolean Algebra are a set of rules that define the properties
of logical operations in Boolean Algebra. These laws are used to simplify
Boolean expressions and to prove the equivalence of different expressions.
The laws of Boolean Algebra are based on the properties of logical
operations such as AND, OR, and NOT.

Boolean algebra's AND operations associates to multiplication in
arithmetic, while OR operations associates to addition. While the
NOT operation is simply the complement of the input.

\newpage

% Table of Laws of Boolean Algebra
% | Laws(2 rows) | Equivalent (2 columns) |
% |    | AND | OR |
\begin{table}[!h]
  \centering
  \begin{tabular}{c c c c}
    \hline
    \textbf{Laws} & \textbf{AND} & \textbf{OR} \\
    \hline
    Identity & $x(1) = x$ & $x + 0 = x$ \\
    Domination & $x(0) = 0$ & $x + 1 = 1$ \\
    Commutative & $xy = yx$ & $x + y = y + x$ \\
    Associative & $x(yz) = (xy)z$ & $x + (y + z) = (x + y) + z$ \\
    Distributive & $x(y + z) = xy + xz$ & $x + yz = (x + y)(x + z)$ \\
    Inverse & $xx' = 0$ & $x + x' = 1$ \\
    Involution (Double Negation) & $(x')' = x$ &  \\
    Idempotent & $xx = x$ & $x + x = x$ \\
    Absorption & $x(x + y) = x$ & $x + xy = x$ \\
    De Morgan's Theorem & $(xy)' = x' + y'$ & $(x + y)' = x'y'$ \\
    \hline
  \end{tabular}
  \caption{Laws of Boolean Algebra}
  \label{table:Laws-Boolean-Algebra}
\end{table}

Table \ref{table:Laws-Boolean-Algebra} shows the laws of Boolean Algebra.
These laws are used to simplify Boolean expressions and to prove the
equivalence of different expressions. The laws are based on the properties
of logical operations such as AND, OR, and NOT.

\noindent\fbox{%
  \parbox{\textwidth}{
    \hspace{0.5em}
    
    \textbf{Exercises}

    \begin{enumerate}
      \item Verify the following laws of Boolean Algebra:
      \begin{enumerate}
        \item Identity Law: $x + 0 = x$
        \item Commutative Law: $xy = yx$
        \item Associative Law: $x(yz) = (xy)z$
        \item Distributive Law: $x(y + z) = xy + xz$
        \item Inverse Law: $xx' = 0$
        \item Involution Law: $(x')' = x$
        \item De Morgan's Theorem: $(xy)' = x' + y'$
      \end{enumerate}
 
      \item Prove the following laws of Boolean Algebra:
      \begin{enumerate}
        \item Domination Law: $x + 1 = 1$
        \item Idempotent Law: $x + x = x$
        \item Absorption Law: $x(x + y) = x$
      \end{enumerate}
    \end{enumerate}

    \hspace{0.5em}
  }
}

\section{Simplifying Boolean Expressions}

Boolean expressions can be simplified using the laws of Boolean Algebra.
Simplifying a Boolean expression involves applying the laws of Boolean
Algebra to reduce the expression to its simplest form. This process
involves combining terms, eliminating redundant terms, and applying the
laws of Boolean Algebra to simplify the expression.

Consider the Boolean expression $f(x, y) = x + x'y$. We can simplify
this expression using the laws of Boolean Algebra as follows:

\begin{align*}
  f(x, y) & = x + x'y \\
  & = (x + x')(x + y) && \text{Distributive Law} \\
  & = (1)(x + y) && \text{Inverse Law} \\
  & = x + y && \text{Identity Law}
  f(x, y) & = x + x'y = x + y
\end{align*}

The expression $f(x, y) = x + x'y$ can be simplified to $f(x, y) = x + y$
using the laws of Boolean Algebra.

Consider the Boolean expression $f(x, y, z) = (x + yz') + (xy)'$. We can
simplify this expression using the laws of Boolean Algebra as follows:

\begin{align*}
  f(x, y, z) & = (x + yz') + (xy)' \\
  & = (x + yz') + x' + y' && \text{De Morgan's Theorem} \\
  & = (x + x') + yz' + y' && \text{Commutative Law} \\
  & = 1 + yz' + y' && \text{Inverse Law} \\
  & = 1 && \text{Domination Law}
\end{align*}

The expression $f(x, y, z) = (x + yz') + (xy)'$ can be simplified to
$f(x, y, z) = 1$ using the laws of Boolean Algebra.

Consider the Boolean expression $f(x, y) = (x + x'y) + (x + y')$. We
can simplify this expression using the laws of Boolean Algebra as follows:

\begin{align*}
  f(x, y) & = (x + x'y) + (x + y') \\
  & = [(x + x')(x + y)] + (x + y') && \text{Distributive Law} \\
  & = [1(x + y)] + (x + y') && \text{Inverse Law} \\
  & = (x + y) + (x + y') && \text{Identity Law} \\
  & = (y + x) + (x + y') && \text{Commutative Law} \\
  & = y + (x + x) + y' && \text{Associative Law} \\
  & = y + x + y' && \text{Idempotent Law} \\
  & = x + y + y' && \text{Commutative Law} \\
  & = x + 1 && \text{Inverse Law} \\
  & = 1 && \text{Domination Law}
\end{align*}

The expression $f(x, y) = (x + x'y) + (x + y')$ can be simplified
to $f(x, y) = 1$ using the laws of Boolean Algebra.

% \begin{align*}
%   f(x, y, z) = [y + x'y + (x + y')]y'
%   & = [(y + x'y) + (x + y')]y' && \text{Associative Law} \\
%   & = [y + (x + y')]y' && \text{Absorption Law} \\
%   & = [(y + y') + x]y' && \text{Commutative Law} \\
%   & = [1 + x]y' && \text{Inverse Law} \\
%   & = (1)y' && \text{Domination Law} \\
%   & = y' && \text{Identity Law}
% \end{align*}

% Exercises
\noindent\fbox{%
  \parbox{\textwidth}{
    \hspace{0.5em}
    
    \textbf{Exercises}

    Simplify the following Boolean expressions using the laws of Boolean
    Algebra:

    \begin{enumerate}
      \item $f(x, y) = (x + y')(x + y)$
      \item $f(w, x) = w + [w + (wx)]$
      \item $f(w, x) = (x' + x')'$
      \item $f(w, x) = (x + x')'$
      \item $f(w, x, y, z) = w + (wx'yz)$
      \item $f(w, x, y, z) = w'(wxyz)'$
      \item $f(x, y, z) = [y + x'y + (x + y')]y'$
      \item $f(w, x, y, z) = wx + (x'z') + (y + z')$
      \item $f(x, y, z) = (x + y)(x + z)$
      \item $f(x, y) = [xy + (x + y)']'$
      \item $f(w, x, y, z) = (w + x' + y + z')y$
      \item $f(x, y, z) = x + y + (x' + y + z)'$
    \end{enumerate}

    \hspace{0.5em}
  }
}

\noindent\fbox{%
  \parbox{\textwidth}{
    \hspace{0.5em}

    % Start at 13
    \begin{enumerate}
      \setcounter{enumi}{12}
      \item $f(x, y, z) = xz + x'y + zy$
      \item $f(x, y, z) = (x + z)(x' + y)(z + y)$
      \item $f(x, y, z) = x' + y' + xyz'$
    \end{enumerate}

    \hspace{0.5em}
  }
}

\section{Principle of Duality}

The \textbf{principle of duality} states that any theorem or identity in
Boolean Algebra remains valid if we interchange the AND and OR operations
and the constants $0$ and $1$. In other words, if we replace each AND
operation with an OR operation, each OR operation with an AND operation,
each $0$ with a $1$, and each $1$ with a $0$, the resulting expression
is also valid.

% Table of Principle of Duality Examples
\begin{table}[!h]
  \centering
  \begin{tabular}{c c c}
    \hline
    \textbf{Expression} & \textbf{Dual Expression} \\
    \hline
    $x + y$ & $xy$ \\
    $x(y + z)$ & $x + yz$ \\
    $x(y + 0)$ & $x + y \cdot 1$ \\
    $x + 1 = 1$ & $x \cdot 0 = 0$ \\
    $x + x = x$ & $x \cdot x = x$ \\
    \hline
  \end{tabular}
  \caption{Examples of the Principle of Duality}
  \label{table:Examples-Principle-Duality}
\end{table}

Table \ref{table:Examples-Principle-Duality} shows examples of the
principle of duality. The dual expression of $x + y$ is $xy$, the dual
expression of $x(y + z)$ is $x + yz$, and so on. The principle of duality
states that any theorem or identity in Boolean Algebra remains valid if
we interchange the AND and OR operations and the constants $0$ and $1$.

% Exercises
\noindent\fbox{%
  \parbox{\textwidth}{
    \hspace{0.5em}
    
    \textbf{Exercises}

    Write the dual expression for the following Boolean expressions:

    \begin{enumerate}
      \item $x + yz$
      \item $x(y + z)$
      \item $x + y + z$
      \item $x(y + z) + x'$
      \item $x + y + z + 1$
      \item $x + x'$
      \item $x + y + z + 0$
      \item $x(y + z) + 1$
      \item $x + y + z + x'$
      \item $x + y + z + x$
    \end{enumerate}

    \hspace{0.5em}
  }
}

\chapter{Logic Gates and Circuits}

\section{Introduction}

\section{Logic Gates and Circuits}

\section{Minimization of Circuits}

\section{Binary Arithmetic and Representation}

\chapter{Graph Theory}

\section{Introduction}

\section{Graphs}

\subsection{Terms and Definitions}

\subsection{Paths and Cycles}

\subsection{Hamiltonian Cycles}

\subsection{Shortest Path Algorithms}

\subsection{Representation of Graphs}

\subsection{Isomorphism of Graphs}

\subsection{Planar Graphs}

\section{Trees}

\subsection{Terms and Definitions}

\subsection{Spanning Trees}

\subsection{Binary Trees}

\subsection{Tree Traversals}

\subsection{Decision Trees}

\subsection{Isomorphism of Trees}

\chapter{Network Models and Petri Nets}

\section{Network Models}

\section{Maximal Flow Algorithm}

\section{Max Flow, Min Cut Theorem}

\section{Matching}

\section{Petri Nets}

\chapter{Automata, Grammars and Languages}

\section{Languages and Grammars}

\section{Finite State Automata}

\section{Regular Expressions}

\chapter{Computational Geometry}

\section{Basics of Computational Geometry}

\section{Closest-Pair Problem}

\section{Convex Hull Algorithm}

\section{Voronoi Diagrams}

\section{Line Segment Intersection}

\section{Applications in Computer Graphics and Geographical Information Systems}

\chapter{References}

\begin{enumerate}[label={\Alph*.}]
  \item \textbf{Books}
    \begin{itemize}
      \item 
    \end{itemize}
  \item \textbf{Other Sources}
    \begin{itemize}
      \item 
    \end{itemize}
\end{enumerate}

\end{document}
